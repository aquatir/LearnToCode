% texstudio
\documentclass[10pt, a4paper]{article}

\usepackage[
	left=1.5cm,
	right=1.5cm,
	top=2cm,
	bottom=2cm,
]{geometry}

\usepackage[T1,T2A]{fontenc}
\usepackage[utf8]{inputenc}
\usepackage[english,russian]{babel}
\usepackage{amsmath}
\usepackage{amsfonts}
\usepackage{amssymb}
\usepackage{makeidx}
\usepackage{listings}
\lstset
{ %Formatting for code 
	language=haskell,
	basicstyle=\footnotesize,
	numbers=left,
	stepnumber=1,
	%frame=single,
	showstringspaces=false,
	tabsize=1,
	breaklines=true,
	breakatwhitespace=false,
}

\setlength{\parindent}{0pt}


\begin{document}
\section{Давайте поговорим о функциях}

\subsection{Как объявлять функции}

У хаскеля есть комилятор $ghc$. Он компилит вашу программу в бинарь

У нас етсь топ левел модуль. Там обычно есть функция main

\begin{lstlisting}[language=haskell]
module Main where
main = putStrLn "Hello, World!"
\end{lstlisting}

Вот эта штука и будет запускаться после компиляции файла. Порядок определений значения не имеет. Можно даже сначала сделать определение функции, а лишь потом объявление.

\subsubsection{Объявление функций}
Никаких лишних скобочек для аргументов функций нет. Например вот функция для прибавления к крестику единицы

\begin{lstlisting}[language=haskell]
f x = x + 1
\end{lstlisting}

а вот для сложения двух чисел

\begin{lstlisting}[language=haskell]
f x y = x + y
\end{lstlisting} 

Функцию можно определить в 2 этапа

\begin{lstlisting}[language=haskell]
f :: Int -> Int
f x = x + 1
\end{lstlisting}

А можно и не определять, так как в хаскеле нормально все с type inference. В таких простых случаях он выводит типы сам. (это 2 примера выше)

Если нужно в мейне вывести больше, чем 1 действие, можно делать так:

\begin{lstlisting}[language=haskell]
main = do 
    print (g 2 4)
    print (f 5)
\end{lstlisting}

\subsubsection{Лямбда функции}
Вообще говоря в хаскеле есть лямбдочки в таком виде

\begin{lstlisting}[language=haskell]
b = \x -> x + 1
\end{lstlisting}

И все вычисления строится именно на лямбдах. Любая функция в хаскеле будет преобразована в такую вот лямбду

В хаскеле есть инфиксные функции. Например

\begin{lstlisting}[language=haskell]
(+++) :: Int -> Int -> Int
x +++ y = x + y + x + y
\end{lstlisting} 

\subsubsection{Функция \$}
Есть интересный оператор \$. У него объявление такое:

\begin{lstlisting}[language=haskell]
($) :: (a -> b) -> a -> b
f $ x = f x
\end{lstlisting}
 
Т.е. это своего рода инфиксные приоритетопонижатель...

Зачем? У него самый низкий приоритет, а у функций (с именем) наоборот самый высокий приоритет.

Это позволяет эквивалентно переписывать конструкции... Например вот так:
\begin{lstlisting}[language=haskell]
main = do 
    print (g 2 4)
    print $ g 2 4
\end{lstlisting}

Получается "возьми функцию слева и примени к ней все что получится из функции (функций) справа", когда та досчитается


\subsubsection{undefined вместо определения}

Можно функцию объявить, но не определить.
\begin{lstlisting}[language=haskell]
funk :: Int -> Int
funk x = undefined
\end{lstlisting}

Оно скомпилится, будет тайп-чекаться везде, но при вызове выкинет жуку.

\subsubsection{Ленивость в хаскеле}

Хаскель ничего не делает, пока ему не нужно. По сему, вы можете определять функции, которые возвращают бесконечные списки... И даже найти их длину!

\begin{lstlisting}[language=haskell]
ones = 1 : ones
main = do
    print $ leng (take 100000 ones)
\end{lstlisting}

Эта штука работает, так как хаскель не пытается вытащить все единички из вызова $ones$, а берет только необходимое количество, чтобы выполнился $take$. К слову, $leng$ тут тоже считает длину сразу же с получением нового элемента из $take$.

\subsection{Стандартные конструкции языка}

\subsubsection{Комментарии}
Однострочечные комментарии - это два минуса. Многострочечные - это фигурная скобка и минус.
\begin{lstlisting}[language=haskell]
-- funk :: Int -> Int one line comment

{- multiple line comment
funk x = undefined
blalbla
-}
\end{lstlisting}


\subsubsection{If'чики}
\begin{lstlisting}[language=haskell]
sign x = 
    if x < 0 
    then -1 
    else 
        if x > 0
        then 1 
        else 0
\end{lstlisting}

ифчики это хорошо, но чаще используют pattern mathing. 

\subsubsection{pattern mathing}
Как пример: функция, которая возвращает строкое представление

\begin{lstlisting}[language=haskell]
data Foo = A | B

fromFoo :: Foo -> String 
fromFoo A = "A"
fromFoo B = "B"
\end{lstlisting}

эта штука на самом деле тоже сахар. Она превращается в 

\begin{lstlisting}[language=haskell]
data Foo = A | B

fromFoo' :: Foo -> String 
fromFoo' x = 
    case x of
      A -> "A"
      B -> "B"
\end{lstlisting}

На самом деле там есть и еще более упоротый синтаксис с фигурными скобками, но так редко пишут... разве что некоторые деды из haskell сообщества.

Паттерн матчинг умный и сам определит, что он объявлен избыточно или недостаточно. При чем во втором случае это может быть и варнинг, и ошибка, в зависимости от параметров компиляции.

\begin{lstlisting}[language=haskell]
goodNumber 1 = True
goodNumber 7 = True
goodNumber x 
    | x < 100 = True
    | x >= 100 = False 
\end{lstlisting}

Тут мы вообще начали объявлять функцию сразу через паттерн матчинг.

\subsubsection{Паттер матчинг, рекурсия и работа со списками}
Циклов в хаскеле нет. Но у нас же есть рекурсия! Вот так, например, можно найти длину списка
\begin{lstlisting}[language=haskell]
leng [] = 0
leng (_:xs) = 1 + leng xs 
\end{lstlisting}

Лист - это на самом деле голова листа + его хвост. В данном случае хвост - это $xs$, а голова - это нечто, но мы его не используем, поэтому вместо головы мы пишем \_.

Последний элемент можно взять, например, вот так:
\begin{lstlisting}[language=haskell]
lastElem [] = error "Oops!"
lastElem (x:[]) = x
lastElem (_:xs) = lastElem xs
\end{lstlisting}

Тут мы список проматчим на 3 значения:
\begin{enumerate}
	\item Пустой список. Для него мы кидаем ошибку (к слову, вот так можно кинуть ошибку)
	\item Список из головы, но без хвоста, т.е. один единственный элементм
	\item Список из головы + хвоста. Здесь голову мы отбрасываемся и опускаемся дальше в хвост
\end{enumerate}

Хаскель компилятор очень хорошо умеет оптимизировать и хвостовую рекурсию, и прочие хвостовые вызовы.

\end{document}
